\styledchapter[Relaciones de Maxwell]{Relaciones de Maxwell}

Son cuatro relaciones entre derivadas parciales que permiten cambiar derivadas en las que interviene la entropía por otras en que intervienen variables tales como $p$, $V$, $T$.  
Existe una relación por cada potencial termodinámico:

\[
\left(\frac{\partial T}{\partial V}\right)_S=-\left(\frac{\partial p}{\partial S}\right)_V 
\quad ;\quad 
\left(\frac{\partial T}{\partial p}\right)_S=\left(\frac{\partial V}{\partial S}\right)_p 
\]
\[
\left(\frac{\partial S}{\partial V}\right)_T=\left(\frac{\partial p}{\partial T}\right)_V 
\quad ;\quad 
\left(\frac{\partial S}{\partial p}\right)_T=-\left(\frac{\partial V}{\partial T}\right)_p
\]

En cuanto a la deducción de estas relaciones, la tarea es bastante sencilla, considerando la diferencial exacta de cada potencial termodinámico.

\section{Diferenciales exactas}

Considerando la forma general de una diferencial exacta:
\begin{equation}
    f(x,y) \rightarrow df = M(x,y)\,dx + N(x,y)\,dy \tag{1.1}
\end{equation}
\begin{equation}
    M = \left( \frac{\partial f}{\partial x}\right)_y  \quad ,\quad 
    N=\left( \frac{\partial f}{\partial y} \right)_x \tag{1.2}
\end{equation}

\section{Deducción de las Relaciones de Maxwell}

\subsection{Primer Caso $(S, V)$}
Usando la diferencial de energía interna:
\[
dU=TdS-pdV
\]
donde $x\rightarrow S$ y $y \rightarrow V$. Entonces:
\[
M=T(S,V), \quad N=-p(S,V)
\]
reemplazando en la relación (1.2):
\begin{equation}
    \boxed{
    \left( \frac{\partial  T}{\partial V} \right)_S = -\left( \frac{\partial p}{\partial  S} \right)_V
    }
\end{equation}

\subsection{Segundo Caso $(p, S)$}
Usando la ecuación diferencial de la entalpía:
\[
dH=TdS+Vdp
\]
donde $x\rightarrow S$ y $y \rightarrow p$. Entonces:
\[
M=T(S,p), \quad N=V(S,p)
\]
reemplazando:
\begin{equation}
    \boxed{
    \left( \frac{\partial T}{\partial p} \right)_S = -\left( \frac{\partial V}{\partial  S} \right)_p
    }
\end{equation}

\subsection{Tercer Caso $(V,T)$}
Usando la ecuación diferencial de energía libre de Helmholtz:
\[
dF=-SdT-pdV
\]
donde $x\rightarrow V$ y $y \rightarrow T$. Entonces:
\[
M=-S(V,T), \quad N=-p(V,T)
\]
reemplazando:
\begin{equation}
    \boxed{
    \left( \frac{\partial  S}{\partial V} \right)_T = \left( \frac{\partial p}{\partial  T} \right)_V
    }
\end{equation}

\subsection{Cuarto Caso $(p,T)$}
Usando la ecuación diferencial de energía libre de Gibbs:
\[
dG=-SdT+Vdp
\]
donde $x\rightarrow p$ y $y \rightarrow T$. Entonces:
\[
M=-S(p,T), \quad N=V(p,T)
\]
reemplazando:
\begin{equation}
    \boxed{
    \left( \frac{\partial  S}{\partial p} \right)_T = -\left( \frac{\partial V}{\partial  T} \right)_p
    }
\end{equation}

