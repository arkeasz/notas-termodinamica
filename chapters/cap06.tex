\styledchapter[Termodinámica Relativista]{Termodinámica Relativista}
\section{Introducción}
La termodinámica relativista surge como una extensión natural de la teoría clásica 
para describir sistemas en los que las velocidades son comparables con la de la luz. 
Este trabajo busca exponer su fundamento histórico y formal, destacando los intentos 
de unificación y las formulaciones modernas más aceptadas.

%----------------------------------------------------
\section{Antecedentes Históricos}
La teoría de la relatividad tuvo profundas implicancias en la termodinámica por dos razones fundamentales. 
En primer lugar, los cuerpos calientes son ligeramente más pesados que los fríos, 
ya que sus átomos o moléculas poseen mayor velocidad y, por tanto, mayor masa efectiva. 
En segundo lugar, debido a que ninguna partícula puede moverse más rápido que la luz, 
la distribución de velocidades de las partículas debe reflejar esa limitación física.

Ambos efectos son, sin embargo, extremadamente pequeños. 
Solo bajo condiciones de temperatura extraordinariamente altas las correcciones relativistas a las fórmulas clásicas adquieren relevancia numérica. Ni siquiera las condiciones del interior solar, con millones de kelvin, son lo bastante extremas. 
En la práctica, los efectos relativistas solo se vuelven significativos en objetos como las enanas blancas, donde además se entrelazan con los efectos cuánticos debido a la alta densidad y la superposición de las longitudes de onda de de Broglie de los electrones libres.

Históricamente, el intento de unificar ambas teorías dio lugar a distintas formulaciones y controversias, entre ellas la conocida como \textit{debate Ott–Planck}, acerca de cómo debe transformarse la temperatura entre sistemas inerciales en movimiento. Mientras que Planck propuso que la temperatura relativista se transforma como $T = T_0 / \gamma$, Ott sugirió la relación opuesta $T = \gamma T_0$, lo que generó una prolongada discusión sobre la correcta interpretación física. Entre los aportes iniciales destacan:
\begin{itemize}
    \item \textbf{Planck (1908):} introdujo la idea de una termodinámica consistente con la relatividad especial.
    \item \textbf{Tolman (1934):} formuló las bases para una teoría de fluidos relativistas.
    \item \textbf{Ott (1963):} reabrió el debate sobre la transformación de la temperatura.
    \item \textbf{Israel y Stewart (1970s):} desarrollaron la teoría moderna de los fluidos disipativos relativistas.
\end{itemize}
Estos trabajos conforman la base conceptual de la termodinámica relativista actual.

%----------------------------------------------------
\section{Fundamento Teórico}

\subsection{Termodinámica Clásica}

La termodinámica clásica estudia el comportamiento macroscópico de los sistemas físicos a través de variables como la energía interna $U$, la entropía $S$, la temperatura $T$, la presión $P$ y el volumen $V$.  
Estas magnitudes describen el estado de equilibrio de un sistema y están relacionadas mediante ecuaciones de estado y las leyes fundamentales de la termodinámica.

El \textbf{primer principio} establece la conservación de la energía:
\[
dU = \delta Q - \delta W,
\]
donde $\delta Q$ representa el calor absorbido por el sistema y $\delta W$ el trabajo realizado por él.  
El \textbf{segundo principio} introduce la noción de entropía $S$ y el carácter irreversible de los procesos naturales:
\[
dS \geq \frac{\delta Q}{T}.
\]
El \textbf{tercer principio}, conocido como postulado de Nernst, indica que la entropía de un sistema perfecto tiende a cero al aproximarse al cero absoluto.

Estas leyes proporcionan una descripción coherente de los procesos energéticos en sistemas macroscópicos bajo el supuesto de que el tiempo y el espacio son absolutos, es decir, independientes del estado de movimiento del observador.  
Este supuesto será el punto donde la Relatividad Especial impone su corrección conceptual.

\subsection{Relatividad Especial}

La Relatividad Especial, formulada por Einstein en 1905, parte del principio de que las leyes de la física son las mismas en todos los sistemas inerciales y que la velocidad de la luz en el vacío, $c$, es constante e independiente del movimiento de la fuente o del observador.  
Estas afirmaciones conducen a la estructura geométrica del espacio–tiempo de Minkowski, descrita por el intervalo invariante:
\[
ds^2 = c^2 dt^2 - dx^2 - dy^2 - dz^2.
\]
Las transformaciones de Lorentz garantizan que este intervalo se conserve entre observadores en movimiento relativo, reemplazando la noción clásica de tiempo absoluto por un tiempo dependiente del estado de movimiento.

En este marco, la energía y el momento forman un cuadrivector $(E/c, \vec{p})$, cuya norma invariante define la masa en reposo:
\[
E^2 = (pc)^2 + (m_0 c^2)^2.
\]
Así, la energía interna de un sistema deja de ser una magnitud puramente escalar y pasa a formar parte del tensor energía–momento $T^{\mu\nu}$, que describe el flujo de energía y momento en el espacio–tiempo.  
La introducción de esta estructura tensorial permitirá expresar las leyes termodinámicas de forma covariante.

\subsection{Compatibilidad entre ambas teorías}

La conciliación entre la termodinámica y la relatividad exige reformular las leyes de conservación y las variables termodinámicas de modo que sean invariantes o covariantes bajo transformaciones de Lorentz.  
En particular, las magnitudes extensivas (como la energía o la entropía total) deben formar parte de tensores o cuadrivectores, mientras que las magnitudes intensivas (como la temperatura o la presión) requieren una definición cuidadosa en sistemas en movimiento.

Uno de los principales desafíos históricos ha sido determinar cómo se transforma la temperatura entre sistemas inerciales.  
Las propuestas de Planck, Einstein, Ott y Landsberg difieren en si la temperatura de un cuerpo en movimiento es menor, mayor o igual a la medida en reposo:
\[
T' = \frac{T}{\gamma}, \quad T' = \gamma T, \quad \text{o bien} \quad T' = T.
\]
Esta discusión, conocida como el \textit{debate Ott–Planck}, refleja la dificultad de definir el equilibrio térmico entre observadores que no comparten el mismo marco de referencia.

Desde un punto de vista moderno, la formulación covariante de la termodinámica relativista se basa en el tensor energía–momento $T^{\mu\nu}$ y en el flujo de entropía $S^{\mu}$, los cuales satisfacen las ecuaciones de conservación:
\[
\nabla_{\mu} T^{\mu\nu} = 0, \qquad \nabla_{\mu} S^{\mu} \geq 0.
\]
Estas expresiones garantizan la conservación de la energía y el momento, así como la irreversibilidad de los procesos, en concordancia con la segunda ley.

En consecuencia, la termodinámica relativista no invalida los principios clásicos, sino que los amplía al exigir su validez en todos los sistemas inerciales, proporcionando una descripción coherente de los procesos térmicos a altas velocidades y altas energías.


%----------------------------------------------------
\section{Formulación Matemática}

En el marco relativista, el estado termodinámico de un sistema continuo se describe mediante el tensor energía–momento \( T^{\mu\nu} \), la densidad de entropía \( s \) y el flujo de materia \( N^{\mu} \).  
Las ecuaciones fundamentales de conservación adoptan la forma covariante:
\[
\nabla_{\mu} T^{\mu\nu} = 0, \qquad \nabla_{\mu} N^{\mu} = 0.
\]
El equilibrio térmico se alcanza cuando el campo de cuatro–velocidad \( u^{\mu} \) y la temperatura \( T \) son tales que el gradiente del potencial de Tolman \( T u^{0} \) permanece constante en el espacio–tiempo.  
Estas relaciones generalizan las leyes clásicas de la termodinámica, garantizando su validez en sistemas en movimiento o bajo campos gravitatorios débiles.

%----------------------------------------------------
\section{Aplicaciones}

La termodinámica relativista encuentra aplicación en numerosos contextos de la física moderna, 
especialmente en sistemas donde las velocidades de las partículas se aproximan a la de la luz 
o donde las densidades de energía son extremadamente elevadas. 
A continuación, se presentan algunos casos representativos que ilustran su importancia.

\subsection{Gas relativista ideal}
La energía cinética de las partículas es comparable con su energía en reposo, lo que ocurre en plasmas o en los primeros instantes del universo.

\subsection{Radiación térmica y presión de radiación}
Es un caso extremo de gas relativista, donde los fotones ejercen una presión que influye en el equilibrio de las estrellas y en el fondo cósmico de microondas

\subsection{Fluidos relativistas en astrofísica y cosmología}
Como los de estrellas de neutrones o el universo temprano, la energía, la presión y el movimiento se describen de forma unificada bajo campos gravitatorios intensos.
