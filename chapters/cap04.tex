\styledchapter[Gran potencial Termodinámico]{Gran Potencial Termodinámico}

El gran potencial se define como:
$$
\Phi_G = U - TS - \mu N
$$

Entonces, su diferencial es:
$$
d\Phi_G = -S\,dT - P\,dV - N\,d\mu
$$

Esta relación se obtiene a partir de:
$$
dU = T\,dS - P\,dV + \mu\,dN
$$

Sustituyendo en la definición de $\Phi_G = U - TS - \mu N$, se obtiene el resultado anterior.

Por tanto, $\Phi_G = \Phi_G(T, V, \mu)$.

Nótese que las variables $T$ y $\mu$ son intensivas, mientras que $\Phi_G$ es extensiva, de modo que debe ser proporcional al volumen:
$$
\Phi_G = V\,\phi_G(T, \mu)
$$

Derivando respecto de $V$ a $T$ y $\mu$ constantes:
$$
\phi_G(T, \mu) = \left( \frac{\partial \Phi_G}{\partial V} \right)_{T, \mu} = -P
$$
por lo tanto:
$$
\Phi_G = -P\,V
$$

Esto es consistente con el hecho de que el potencial químico $\mu$ es la energía libre de Gibbs por partícula:
$$
\mu N = G = U - TS + P V
$$
y, en consecuencia,
$$
P V = - (U - TS - \mu N) = -\Phi_G.
$$
