

\styledchapter[Gran Potencial Termodinámico]{Gran Potencial Termodinámico}

\section{Definición del Gran Potencial}

El gran potencial se define como:
\begin{equation}
\Phi_G = U - TS - \mu N \tag{4.1}
\end{equation}

Entonces, su diferencial es:
\begin{equation}
d\Phi_G = -S\,dT - P\,dV - N\,d\mu \tag{4.2}
\end{equation}

Esta relación se obtiene a partir de:
\begin{equation}
dU = T\,dS - P\,dV + \mu\,dN \tag{4.3}
\end{equation}

Sustituyendo en la definición de $\Phi_G = U - TS - \mu N$, se obtiene el resultado anterior.

\section{Dependencia funcional y naturaleza extensiva}

Por tanto, $\Phi_G = \Phi_G(T, V, \mu)$.

Nótese que las variables $T$ y $\mu$ son intensivas, mientras que $\Phi_G$ es extensiva, de modo que debe ser proporcional al volumen:
\begin{equation}
\Phi_G = V\,\phi_G(T, \mu) \tag{4.4}
\end{equation}

\section{Relación entre el gran potencial y la presión}

Derivando respecto de $V$ a $T$ y $\mu$ constantes:
\begin{equation}
\phi_G(T, \mu) = \left( \frac{\partial \Phi_G}{\partial V} \right)_{T, \mu} = -P \tag{4.5}
\end{equation}
por lo tanto:
\begin{equation}
\Phi_G = -P\,V \tag{4.6}
\end{equation}

\section{Relación con la energía libre de Gibbs}

Esto es consistente con el hecho de que el potencial químico $\mu$ es la energía libre de Gibbs por partícula:
\begin{equation}
\mu N = G = U - TS + PV \tag{4.7}
\end{equation}
y, en consecuencia:
\begin{equation}
PV = - (U - TS - \mu N) = -\Phi_G.
\end{equation}
