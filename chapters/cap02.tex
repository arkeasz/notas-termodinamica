\styledchapter[Ciclo de Carnot]{Ciclo de Carnot}

Es un proceso cíclico reversible que utiliza un gas ideal, y que consta de dos transformaciones isotérmicas y dos adiabáticas.

\textbf{Máquina de Carnot}
En una máquina de carnot el ciclo se recorre en sentido horario para que el gas produzca trabajo siempre que la sustancia del trabajo sea un gas ideal


\begin{figure}[h]
    % \includegraphics[width=0.5\linewidth]{maquina_de_carnot.png}
    \centering
    \caption{Máquina de Carnot}
    % \label{Máquina de Carnot}
\end{figure}

\begin{description}
    \item[$1\rightarrow 2$ Expansión Isoterma]{El gas absorbe una cantidad de calor $Q_1$, manteniéndose a la temperatura del foco caliente $T_H$}
    \item[$2\rightarrow 3$ Expansión Adiabática]{El gas se enfría sin pérdida de calor hasta la temperatura del foco frío $T_C$} 
    \item[$3\rightarrow 4$ Compresión Isoterma]{El gas cede el calor $Q_2$ al foco frío sin variar la temperatura.} 
    \item[$4\rightarrow 1$ Compresión Adiabática]{El gas se calienta hasta la temperatura del foco caliente $T_H$, cerrando el ciclo.} 
\end{description}


\textbf{Refrigerador de Carnot}
A diferencia de la máquina de Carnot, el ciclo se recorre en sentido antihorario, ya que el trabajo es negativo, puesto que, es consumido por el gas.


\begin{figure}[h]
    \centering
    % \includegraphics[width=0.5\linewidth]{refrigerador_de_carnot.png}
    \caption{Refrigerador de Carnot}
    \label{fig:placeholder}
\end{figure}

\begin{description}
    \item[$1\rightarrow 2$ Expansión Adiabática]{El gas se enfría sin pérdida de calor hasta la temperatura del foco frío $T_C$} 
    \item[$2\rightarrow 3$ Expansión Isoterma]{El gas se mantiene a la temperatura del foco frío $T_C$ y durante la expansión, absorbe el calor $Q_2$ de dicho foco}
    \item[$3\rightarrow 4$ Compresión Adiabática]{El gas se calienta hasta la temperatura del foco caliente $T_H$, sin intercambio de calor.} 
    \item[$4\rightarrow 1$ Compresión Isoterma]{El gas cede el calor $Q_1$ al foco caliente manteniéndose a la temperatura de dicho foco $T_H$ y cerrando el ciclo.} 
\end{description}
