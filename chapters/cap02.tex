\styledchapter[Ciclo de Carnot]{Ciclo de Carnot}

El ciclo de Carnot es un proceso cíclico reversible que utiliza un gas ideal, compuesto por dos transformaciones isotérmicas y dos adiabáticas.  
Representa el ciclo teórico más eficiente posible entre dos fuentes de calor que operan a temperaturas distintas.

\section{Máquina de Carnot}

En una máquina de Carnot el ciclo se recorre en sentido horario, de manera que el gas realiza trabajo neto.  
El principio fundamental es que todo el calor absorbido del foco caliente se convierte parcialmente en trabajo, y el resto se cede al foco frío.

\begin{figure}[h]
    \centering
    % \includegraphics[width=0.5\linewidth]{figures/maquina_de_carnot.png}
    \caption{Esquema de una máquina de Carnot.}
    \label{fig:maquina-carnot}
\end{figure}

El ciclo se describe con las siguientes etapas:

\begin{description}
    \item[$1\rightarrow 2$ Expansión Isotérmica]  
    El gas absorbe una cantidad de calor $Q_1$, manteniéndose a la temperatura del foco caliente $T_H$. El volumen aumenta y se realiza trabajo sobre el entorno.

    \item[$2\rightarrow 3$ Expansión Adiabática]  
    El gas continúa expandiéndose sin intercambio de calor ($Q=0$), enfriándose hasta alcanzar la temperatura del foco frío $T_C$.

    \item[$3\rightarrow 4$ Compresión Isotérmica]  
    El gas cede calor $Q_2$ al foco frío a temperatura constante $T_C$, mientras el volumen disminuye.

    \item[$4\rightarrow 1$ Compresión Adiabática]  
    El gas se comprime sin intercambio de calor ($Q=0$), aumentando su temperatura hasta recuperar $T_H$ y cerrando el ciclo.
\end{description}

La eficiencia térmica de la máquina de Carnot se define como:
\[
\eta = 1 - \frac{T_C}{T_H}
\]
donde las temperaturas se expresan en Kelvin.  
Este límite representa el máximo rendimiento posible de cualquier máquina térmica ideal entre dos focos.

\section{Refrigerador de Carnot}

El refrigerador de Carnot opera con el mismo principio, pero el ciclo se recorre en sentido antihorario.  
En este caso, el gas consume trabajo externo para transferir calor desde el foco frío hacia el foco caliente.

\begin{figure}[h]
    \centering
    % \includegraphics[width=0.5\linewidth]{figures/refrigerador_de_carnot.png}
    \caption{Refrigerador de Carnot.}
    \label{fig:refrigerador-carnot}
\end{figure}

Las etapas del ciclo son las siguientes:

\begin{description}
    \item[$1\rightarrow 2$ Expansión Adiabática]  
    El gas se enfría al expandirse sin intercambio de calor hasta la temperatura del foco frío $T_C$.

    \item[$2\rightarrow 3$ Expansión Isotérmica]  
    El gas, a temperatura constante $T_C$, absorbe el calor $Q_2$ del foco frío.

    \item[$3\rightarrow 4$ Compresión Adiabática]  
    El gas se comprime sin intercambio de calor, elevando su temperatura hasta $T_H$.

    \item[$4\rightarrow 1$ Compresión Isotérmica]  
    El gas cede el calor $Q_1$ al foco caliente, manteniéndose a temperatura $T_H$, y el ciclo se repite.
\end{description}

El coeficiente de rendimiento (COP) de un refrigerador de Carnot se expresa como:
\[
COP_{ref} = \frac{T_C}{T_H - T_C}
\]
lo que indica la relación entre el calor extraído del foco frío y el trabajo consumido.

\section{Importancia del Ciclo de Carnot}

El ciclo de Carnot es una referencia teórica: ningún motor real puede superarlo en eficiencia.  
Toda desviación práctica respecto al rendimiento ideal se debe a irreversibilidades, fricciones, pérdidas térmicas y no idealidades del gas.

Por eso, el análisis de Carnot no solo establece un límite, sino que también ofrece un criterio fundamental para evaluar la calidad de los sistemas térmicos reales.
