\styledchapter[Generalización de las leyes de la Termodinámica para sistemas abiertos]
{Generalización de las leyes \\ de la Termodinámica \\ \large Para sistemas abiertos}

\begin{center}
 La generalización de las leyes, en particular la Primera Ley, se traduce en una ecuación de balance de energía que considera la energía que entra y sale del sistema con el flujo de masa, el calor y el trabajo. Esto contrasta con los sistemas cerrados, donde solo se consideran los flujos de calor y trabajo. Al incorporar el transporte de energía asociado al movimiento de la materia, esta aproximación permite analizar estados estacionarios y transitorios, proporcionando una base robusta para la descripción de procesos que implican movimiento y transferencia de masa, como los que se encuentran en plantas químicas y centrales eléctricas.
\end{center}
Por primera ley tenemos que 
\begin{equation}
    \delta  Q=dU+pdV-\mu dN \tag{3.1}
\end{equation}
donde:
\begin{multicols}{2}

\begin{itemize}
    \item Q = Calor
    \item U = Energía interna
    \item p = presión
    \item V = Volumen
    \item $\mu$ = potencial químico
    \item N = número de componentes del sistema
\end{itemize}
    
\end{multicols}

% \begin{figure}[h]
%     \centering
%     \includegraphics[width=0.5\linewidth]{sistema_1.png}
%     % \caption{Enter Caption}
%     \label{fig:placeholder}
% \end{figure}
Donde $dE$ es el elemento infinitesimal de la energía interna que sale del sistema.
Tenemos que $$dE=\mu dN$$ donde $\mu$ juega el rol de una función constante como multiplicador de lagrange.
Luego: \begin{equation}
    \delta Q= dU + \delta W -\mu dN \tag{3.2}
\end{equation}Forma diferencial de la primera ley de la termodinámica para un sistema abierto y un monocomponente.
Para la segunda ley de la termodinámica tenemos:
% \begin{figure}[h]
%     \centering
%     \includegraphics[width=0.5\linewidth]{sistema_2.png}
%     \caption{Enter Caption}
%     \label{fig:placeholder}
% \end{figure} es así que $$dS = \lambda \delta Q \xrightarrow{} \lambda = \frac{1}{T}$$reemplazando tendríamos
\begin{equation}
    \delta  Q =  TdS \tag{3.3}
\end{equation}Siendo la forma diferencial de la segunda ley de la termodinámica para un sistema abierto y monocomponente.
Igualando la ec. (3.1) con la ec. (3.3) tenemos:
\begin{equation}
    TdS  = dU + pdV - \mu dN \tag{3.4}
\end{equation}
El método de potenciales termodinámicos en la teoría clásica del calor  (termodinámica) se basa en la solución de la ecuación fundamental de la termodinámica (3.4).

Analizaremos distintos casos usando diferentes variables independientes, esto es lo que llamaremos casos.

\begin{description}
    \item[\textbf{Primer Caso}  $(S,V,N)$] {
        entonces 
        \begin{equation}
            dU = TdS - pdV + \mu dN  \tag{3.5}
        \end{equation}
        \begin{equation}
             U = U(S,V,N) \tag{3.6}
        \end{equation}
siendo la ecuación de arriba el potencial termodinámico de la energía interna.
        Por otro lado se cumple también:
        \begin{equation}
            dU = \left( \frac{\partial U}{\partial S} \right)_{V,N}dS +
            \left( \frac{\partial U}{\partial V} \right)_{S,N}dV +
            \left( \frac{\partial U}{\partial N} \right)_{S,V}dN \tag{3.7}
        \end{equation}
Observamos que tanto la ecuación (3.5) como la ecuación (3.7) representan la misma esencia matemática. Ambas ecuaciones pudiéndose igualar, obteniendo:
        \begin{multicols}{3}
                Temperatura \\[4pt]
                $T=\left( \frac{\partial U}{\partial S} \right)_{V,N}$
                \columnbreak \\
                Presión \\[4pt]
                $p=-\left( \frac{\partial U}{\partial V} \right)_{S,N}$
                \columnbreak  \\
                Potencial Químico \\[4pt]
                $\mu=\left( \frac{\partial U}{\partial N} \right)_{S,V}$
        \end{multicols}
    }
    \item [\textbf{Segundo Caso  $(T,V,N)$}] {
    entonces tomando una función $F$ de tal manear que $F=F(V,T,N$ es nuestra función hipótesis. Por tanto:
    \begin{equation}
        dF=\left( \frac{\partial F}{\partial V} \right)_{T,N}dV + \left( \frac{\partial F}{\partial T} \right)_{V,N}dT + \left( \frac{\partial F}{\partial N} \right)_{V,T}dN \tag{3.8}
    \end{equation}siendo la ecuación fundamental de la termodinámica hipotética.
    Usando el recurso matemático $TdS = d[TS] -  SdT$ y reemplazándolo en la ec. (3.8), tenemos:
    \begin{equation}
        d[TS] - SdT = dU +pdV - \mu dN \tag{3.9}
    \end{equation}
    \begin{equation}
        d[U-TS] = -pdV - SdT + \mu dN \tag{3.10}
    \end{equation}
    }
\end{description}